\documentclass[12pt,twoside]{article}
\usepackage[none]{hyphenat}
\usepackage[english]{babel}
\usepackage{amsmath,amssymb}
\usepackage{graphicx}
\usepackage{listings}
\usepackage{caption}
\usepackage{hyperref}
\usepackage{booktabs}
\usepackage{array}
\usepackage{geometry}
\usepackage{fancyhdr}
\usepackage{xcolor}
\geometry{a4paper, margin=1in, top=0.8in}
\lstset{
  frame=single,
  breaklines=true
}


\newcommand{\mydet}[1]{\ensuremath{\begin{vmatrix}#1\end{vmatrix}}}
\providecommand{\brak}[1]{\ensuremath{\left(#1\right)}}
\providecommand{\norm}[1]{\left\lVert#1\right\rVert}
\newcommand{\solution}{\noindent \textbf{\textcolor{customblue}{Solution:} }}
\newcommand{\myvec}[1]{\ensuremath{\begin{pmatrix}#1\end{pmatrix}}}
\let\vec\mathbf


\definecolor{customblue}{HTML}{2596BE}


\pagestyle{fancy}
\fancyhf{} 
\renewcommand{\headrulewidth}{0pt} 


\fancyhead[RO]{\textcolor{customblue}{31}}
\fancyhead[LE]{\textcolor{customblue}{32}}


\fancyhead[LO]{\textcolor{customblue}{POLYNOMIALS}}
\fancyhead[RE]{\textcolor{customblue}{MATHEMATICS}}

\makeatletter
\def\headrule{%
  {\color{customblue}\hrule\@height\headrulewidth\@width\textwidth \vskip-\headrulewidth}%%
}
\makeatother
\renewcommand{\headrulewidth}{0.5pt}


\begin{document}


\section*{\textcolor{customblue}{Example 4:}}

Find a quadratic polynomial, the sum and product of whose zeroes are $-3$ and $2$, respectively.

\solution Let the quadratic polynomial be $ax^2 + bx + c$, and its zeroes be $\alpha$ and $\beta$. \\
We have
\[\alpha + \beta = -3 = \frac{-b}{a},\]
and \[\alpha\beta=2=\frac{c}{a}\]
\\If $a = 1$, then $b=3$ and $c=2.$
\\So, one quadratic polynomial which fits the given conditions is:
$x^2 + 3x + 2.$

You can check that any other quadratic polynomial that fits these conditions will be of the form:
$k(x^2 + 3x + 2), \text {where \(k\) is real.}$

Let us now look at cubic polynomials. Do you think a similar relation holds
between the zeroes of a cubic polynomial and its coefficients?

Let us consider $p(x)=2x^3-5x^2-14x+8.$

You can check that $p(x)=0$ for x=4,-2,$\frac{1}{2}$. Since $p(x)$ can have almost three zeroes, these are the zeroes of $2x^3-5x^2-14x+8.$ Now,


\quad \text{sum of the zeroes:}
\[4 + (-2) + \frac{1}{2} = \frac{1}{2} = -\frac{-5}{2} = -\frac{\text{Coefficient of } x^2}{\text{Coefficient of } x^3}\]

\quad \text{product of zeroes:}
\[4 \cdot (-2) \cdot \frac{1}{2} = -4 = \frac{-8}{2} = \frac{-\text{Constant term}}{\text{Coefficient of } x^3}\]

However, there is one more relationship here. Consider the sum of the products
of the zeroes taken two at a time. We have


\[4 (-2) + (-2) \cdot \frac{1}{2} + 4 \cdot \frac{1}{2} = -8 -1 + 2 = -7 = \frac{-14}{2} = -\frac{\text{Coefficient of } x}{\text{Coefficient of } x^3}\]
In general, it can be proved that if $\alpha, \beta, \gamma$ are the zeroes of the cubic polynomial $ax^3 + bx^2 + cx + d$, then:
\[\alpha + \beta + \gamma = -\frac{b}{a},\]


\[\alpha\beta + \beta\gamma + \gamma\alpha = \frac{c}{a},\]


\[\alpha\beta\gamma = -\frac{d}{a}\]

\section*{\textcolor{customblue}{Example 5:}}
Verify that $3$, $-1$, and $\frac{1}{3}$ are the zeroes of the cubic polynomial $p(x) = 3x^3 - 5x^2 - 11x - 3$, and then verify the relationship between the zeroes and the coefficients.

\solution Comparing the given polynomial with $ax^3 + bx^2 + cx + d$, we get:

a = 3,b = -5,c = -11,d = -3. Further

\(p(3) = 3(3)^3 - 5(3)^2 - 11(3) - 3 = 81 - 45 - 33 - 3 = 0\)


\((-1) = 3(-1)^3 - 5(-1)^2 - 11(-1) - 3 = -3 - 5 + 11 - 3 = 0\)


\(p\left(\frac{1}{3}\right) = 3\left(\frac{1}{3}\right)^3 - 5\left(\frac{1}{3}\right)^2 - 11\left(\frac{1}{3}\right) - 3,\) 

\[=\frac{1}{9} - \frac{5}{9} - \frac{11}{3} - 3 = -\frac{2}{3}+\frac{2}{3}= 0\]

Therefore, $3$, $-1$, and $\frac{1}{3}$ are the zeroes of $3x^3 - 5x^2 - 11x - 3$. 

So we take: $\alpha = 3,$ $ \beta = -1 $ $\text{and}  $ $\gamma = \frac{1}{3}$

Now,


\quad \(\alpha + \beta + \gamma = 3 + (-1) + -\frac{1}{3} = 2-\frac{1}{3} = \frac{5}{3}=\frac{-(-5)}{3}=\frac{b}{a}\)


\quad \(\alpha\beta + \beta\gamma + \gamma\alpha = 3  (-1) + (-1) (-\frac{1}{3}) + -(\frac{1}{3})  3 = -3 + \frac{1}{3} - 1 = -\frac{11}{3} = \frac{c}{a} \)


\quad \(\alpha\beta\gamma = 3 (-1)  (-\frac{1}{3}) = 1  = \frac{-(-3)}{3}=\frac{-d}{a}\)

\end{document}

