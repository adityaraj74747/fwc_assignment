\documentclass{article}
\usepackage{amsmath}
\usepackage{array}
\usepackage{geometry}
\geometry{margin=1in}

\begin{document}

\section*{Deduction of \(f = r \oplus p \oplus q\)}

\subsection*{Truth Table}

\[
\begin{array}{|c|c|c|c|}
\hline
r & p & q & f \\
\hline
0 & 0 & 0 & 0 \\
0 & 0 & 1 & 1 \\
0 & 1 & 0 & 1 \\
0 & 1 & 1 & 0 \\
1 & 0 & 0 & 1 \\
1 & 0 & 1 & 0 \\
1 & 1 & 0 & 0 \\
1 & 1 & 1 & 1 \\
\hline
\end{array}
\]

\subsection*{Step 1: Write the canonical sum of products (SOP)}

From the truth table, \(f=1\) for minterms where

\[
(r, p, q) = (0,0,1), (0,1,0), (1,0,0), (1,1,1)
\]

which correspond to minterms:

\[
m_1, m_2, m_4, m_7
\]

Therefore, the SOP expression is:

\[
f = r' p' q + r' p q' + r p' q' + r p q
\]

\subsection*{Step 2: Recognize the XOR pattern}

The above expression matches the standard 3-variable XOR:

\[
f = r \oplus p \oplus q
\]

since \(f=1\) when an odd number of inputs are 1.

\subsection*{Step 3: Final expression}

\[
\boxed{
f = r \oplus p \oplus q
}
\]

\bigskip

\section*{Raspberry Pi Pico Pin Connections}

\begin{tabular}{|c|c|c|}
\hline
\textbf{Signal} & \textbf{Pin Number} & \textbf{Description} \\
\hline
\(r\) & GPIO 14 & Push button input with pull-down resistor \\
\(p\) & GPIO 15 & Push button input with pull-down resistor \\
\(q\) & GPIO 16 & Push button input with pull-down resistor \\
\(f\) (Output) & GPIO 17 & LED output \\
\hline
\end{tabular}

\bigskip
\textit{Note: Connect push buttons between the respective GPIO pins and 3.3V with internal pull-down enabled. Connect an LED (with suitable resistor) to GPIO 17 for output indication.}

\end{document}
