\documentclass[12pt]{article}
\usepackage{amsmath, amssymb}
\usepackage{geometry}
\usepackage{fancyhdr}
\pagestyle{fancy}
\geometry{margin=1in}
\fancyhead[L]{}
\fancyhead[R]{}
\renewcommand{\headrulewidth}{0.4pt}

\begin{document}

\section*{Problem Setup}

Given a 4:1 multiplexer:
\begin{itemize}
    \item Select lines: $a$ and $b$
    \item Inputs:
    \[
    \begin{aligned}
    i_0 &= c \\
    i_1 &= d \\
    i_2 &= \overline{c} \\
    i_3 &= \overline{c} \cdot \overline{d}
    \end{aligned}
    \]
\end{itemize}

\section*{Step 1: MUX Output Expression}

\[
f(a,b,c,d) = \overline{a}\,\overline{b}\,c + \overline{a}b\,d + a\,\overline{b}\,\overline{c} + ab\,\overline{c}\,\overline{d}
\]

\section*{Step 2: Expand Terms to Include All Variables}

\begin{align*}
\text{Term 1: } & \overline{a}\,\overline{b}\,c = \overline{a}\,\overline{b}\,c(d + \overline{d}) = \overline{a}\,\overline{b}\,c\,d + \overline{a}\,\overline{b}\,c\,\overline{d} \\
\text{Term 2: } & \overline{a}b\,d = \overline{a}b\,d(c + \overline{c}) = \overline{a}b\,d\,c + \overline{a}b\,d\,\overline{c} \\
\text{Term 3: } & a\,\overline{b}\,\overline{c} = a\,\overline{b}\,\overline{c}(d + \overline{d}) = a\,\overline{b}\,\overline{c}\,d + a\,\overline{b}\,\overline{c}\,\overline{d} \\
\text{Term 4: } & ab\,\overline{c}\,\overline{d} \text{ already includes all variables}
\end{align*}

\section*{Step 3: List All Resulting Minterms}

Using variable order $a, b, c, d$ (MSB to LSB):

\begin{itemize}
    \item $\overline{a}\,\overline{b}\,c\,d \rightarrow 0011 = m_3$
    \item $\overline{a}\,\overline{b}\,c\,\overline{d} \rightarrow 0010 = m_2$
    \item $\overline{a}b\,d\,c \rightarrow 0111 = m_7$
    \item $\overline{a}b\,d\,\overline{c} \rightarrow 0101 = m_5$
    \item $a\,\overline{b}\,\overline{c}\,d \rightarrow 1001 = m_9$
    \item $a\,\overline{b}\,\overline{c}\,\overline{d} \rightarrow 1000 = m_8$
    \item $a\,b\,\overline{c}\,\overline{d} \rightarrow 1100 = m_{12}$
\end{itemize}

\section*{Final Answer: Sum of Minterms}

\[
\boxed{f(a,b,c,d) = \sum m(2, 3, 5, 7, 8, 9, 12)}
\]

\section*{Raspberry Pi Pico Pin Connections}

\begin{tabular}{|c|c|c|}
\hline
\textbf{Signal} & \textbf{GPIO Pin} & \textbf{Purpose} \\
\hline
\(a\) & GPIO 14 & Select input A (push button with pull-down) \\
\(b\) & GPIO 15 & Select input B (push button with pull-down) \\
\(c\) & GPIO 16 & Data input (push button with pull-down) \\
\(d\) & GPIO 18 & Data input (push button with pull-down) \\
\(f\) & GPIO 17 & Output to LED \\
\hline
\end{tabular}

\bigskip

\textit{Note: Connect each button between the GPIO and 3.3V. Enable internal pull-down resistors. Use a current-limiting resistor with the LED from GPIO 17 to GND.}

\end{document}
