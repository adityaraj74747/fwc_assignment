\documentclass{article}
\usepackage{amsmath}
\usepackage{array}
\usepackage{geometry}
\geometry{margin=1in}

\begin{document}

\section*{Minterm Expression of \(f(p, q, r) = pq + qr' + pr'\)}

\subsection*{Step 1: Truth Table}

\[
\begin{array}{|c|c|c|c|c|c|c|}
\hline
p & q & r & pq & qr' & pr' & f = pq + qr' + pr' \\
\hline
0 & 0 & 0 & 0 & 0 & 0 & 0 \\
0 & 0 & 1 & 0 & 0 & 0 & 0 \\
0 & 1 & 0 & 0 & 1 & 0 & 1 \\
0 & 1 & 1 & 0 & 0 & 0 & 0 \\
1 & 0 & 0 & 0 & 0 & 1 & 1 \\
1 & 0 & 1 & 0 & 0 & 0 & 0 \\
1 & 1 & 0 & 1 & 1 & 1 & 1 \\
1 & 1 & 1 & 1 & 0 & 0 & 1 \\
\hline
\end{array}
\]

\subsection*{Step 2: Identify Rows Where \(f = 1\)}

The function is true for the following combinations:
\[
\begin{aligned}
(0,1,0) &\Rightarrow m_2 = p' q r' \\
(1,0,0) &\Rightarrow m_4 = p q' r' \\
(1,1,0) &\Rightarrow m_6 = p q r' \\
(1,1,1) &\Rightarrow m_7 = p q r \\
\end{aligned}
\]

\subsection*{Step 3: Canonical SOP (Minterm) Expression}

\[
\boxed{f(p, q, r) = m_2 + m_4 + m_6 + m_7}
\]



\section*{Raspberry Pi Pico Pin Connections}

\begin{tabular}{|c|c|c|}
\hline
\textbf{Signal} & \textbf{GPIO Pin} & \textbf{Description} \\
\hline
\(p\) & GPIO 15 & Push button input with pull-down resistor \\
\(q\) & GPIO 16 & Push button input with pull-down resistor \\
\(r\) & GPIO 14 & Push button input with pull-down resistor \\
\(f\) (Output) & GPIO 17 & Output to LED \\
\hline
\end{tabular}

\bigskip
\textit{Note: Connect push buttons between GPIO pins and 3.3V. Enable internal pull-down resistors. Connect LED (with resistor) from GPIO 17 to GND.}

\end{document}
