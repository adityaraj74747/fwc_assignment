\documentclass{article}
\usepackage{amsmath}
\usepackage{geometry}
\geometry{margin=1in}

\begin{document}

\large  % Slightly larger font than normal

\section*{Minimization of the Boolean Function}

Given the K-map with variables \(x, y, z\):

\[
\begin{array}{c|cccc}
x \backslash yz & 00 & 01 & 11 & 10 \\ \hline
0 & 1 & 1 & 1 & 0 \\
1 & 0 & 0 & 1 & 0 \\
\end{array}
\]

where the entries with 1 correspond to the minterms:

\[
m_0, m_1, m_3, m_7
\]

---

\textbf{Step 1: Write the canonical sum of products}

\[
f = m_0 + m_1 + m_3 + m_7 = x'y'z' + x'y'z + x'yz + xyz
\]

---

\textbf{Step 2: Group terms from the K-map}

From the K-map, group the first three 1's in the top row:

\[
x'(y'z' + y'z + y z) = x' (y' + y z)
\]

Using Boolean algebra, simplify inside the parentheses:

\[
y' + y z = y' + z
\]

Thus,

\[
f = x'(y' + z) + x y z
\]

---

\textbf{Step 3: Expand the expression}

\[
f = x'y' + x'z + xyz
\]

---

\textbf{Step 4: Check for further simplification}

Using the consensus theorem, the term \(x'z\) is redundant because it is the consensus of \(x'y'\) and \(y z\). Hence, we can write:

\[
f = x'y' + y z
\]

---

\textbf{Final simplified expression:}

\[
\boxed{
f = x'y' + y z
}
\]

\end{document}
