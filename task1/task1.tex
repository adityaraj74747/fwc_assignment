\documentclass[12pt,twoside]{article}
\usepackage[none]{hyphenat}
\usepackage[english]{babel}
\usepackage{amsmath,amssymb}
\usepackage{graphicx}
\usepackage{listings}
\usepackage{caption}
\usepackage{hyperref}
\usepackage{booktabs}
\usepackage{array}
\usepackage{geometry}
\usepackage{fancyhdr}
\usepackage{xcolor}
\geometry{a4paper, margin=1in, top=0.8in}
\lstset{
  frame=single,
  breaklines=true
}

% Custom macros
\newcommand{\mydet}[1]{\ensuremath{\begin{vmatrix}#1\end{vmatrix}}}
\providecommand{\brak}[1]{\ensuremath{\left(#1\right)}}
\providecommand{\norm}[1]{\left\lVert#1\right\rVert}
\newcommand{\solution}{\noindent \textbf{\textcolor{customblue}{Solution:} }}
\newcommand{\myvec}[1]{\ensuremath{\begin{pmatrix}#1\end{pmatrix}}}
\let\vec\mathbf

% Define custom color
\definecolor{customblue}{HTML}{2596BE}

% Fancy header setup with different headers for odd/even pages
\pagestyle{fancy}
\fancyhf{} % Clear header/footer
\renewcommand{\headrulewidth}{0pt} % Remove default line

% Odd (right) pages
\fancyhead[RO]{\textcolor{customblue}{MATHEMATICS}}
\fancyhead[LE]{\textcolor{customblue}{POLYNOMIALS}}

% Even (left) pages
\fancyhead[LO]{\textcolor{customblue}{28}}
\fancyhead[RE]{\textcolor{customblue}{29}}

\makeatletter
\def\headrule{%
  {\color{customblue}\hrule\@height\headrulewidth\@width\textwidth \vskip-\headrulewidth}%%
}
\makeatother
\renewcommand{\headrulewidth}{0.5pt}

\begin{document}

\begin{center}
\textbf{\Large \textcolor{customblue}{CHAPTER-2 \\ POLYNOMIALS}}
\end{center}

\begin{figure}[h!]
\centering
\includegraphics[width=0.8\linewidth]{imgg.png}
\caption*{\textcolor{customblue}{\textbf{Fig 2.10}}}
\end{figure}


\section*{\textcolor{customblue}{Example 4:}}

Find a quadratic polynomial, the sum and product of whose zeroes are $-3$ and $2$, respectively.

\solution Let the quadratic polynomial be $ax^2 + bx + c$, and its zeroes be $\alpha$ and $\beta$. \\
We have: 
\[\alpha + \beta = -3, \quad \alpha\beta = 2\]
Assuming $a = 1$, then:
\[b = -(\alpha + \beta) = 3, \quad c = \alpha\beta = 2\]
So, one quadratic polynomial which fits the given conditions is:
\[x^2 + 3x + 2\]
You can check that any other quadratic polynomial that fits these conditions will be of the form:
\[k(x^2 + 3x + 2), \quad \text{where } k \in \mathbb{R}\]

\section*{\textcolor{customblue}{Cubic Polynomial Example}}

Let us now look at cubic polynomials. Do you think a similar relation holds between the zeroes of a cubic polynomial and its coefficients?

Let us consider:
\[p(x) = 2x^3 - 5x^2 - 14x + 8\]

You can check that $p(x) = 0$ for $x = 4, -2, \frac{1}{2}$. Since $p(x)$ can have at most three zeroes, these are the zeroes of $2x^3 - 5x^2 - 14x + 8$. Now,

\textbf{Sum of zeroes:}
\[4 + (-2) + \frac{1}{2} = \frac{1}{2} = -\frac{-5}{2} = -\frac{\text{Coefficient of } x^2}{\text{Coefficient of } x^3}\]

\textbf{Product of zeroes:}
\[4 \cdot (-2) \cdot \frac{1}{2} = -4 = \frac{-8}{2} = \frac{-\text{Constant term}}{\text{Coefficient of } x^3}\]

\textbf{Sum of products of zeroes taken two at a time:}
\[4 \cdot (-2) + (-2) \cdot \frac{1}{2} + 4 \cdot \frac{1}{2} = -8 -1 + 2 = -7 = \frac{-14}{2} = -\frac{\text{Coefficient of } x}{\text{Coefficient of } x^3}\]

\noindent In general, it can be proved that if $\alpha, \beta, \gamma$ are the zeroes of the cubic polynomial:
\[ax^3 + bx^2 + cx + d\]
Then:
\[\alpha + \beta + \gamma = -\frac{b}{a}, \quad \alpha\beta + \beta\gamma + \gamma\alpha = \frac{c}{a}, \quad \alpha\beta\gamma = -\frac{d}{a}\]

\section*{\textcolor{customblue}{Example 5:}}

Verify that $3$, $-1$, and $\frac{1}{3}$ are the zeroes of the cubic polynomial $p(x) = 3x^3 - 5x^2 - 11x - 3$, and then verify the relationship between the zeroes and the coefficients.

\solution Comparing the given polynomial with $ax^3 + bx^2 + cx + d$, we get:
\[a = 3,\quad b = -5,\quad c = -11,\quad d = -3\]

Now we check each zero:
\[p(3) = 3(3)^3 - 5(3)^2 - 11(3) - 3 = 81 - 45 - 33 - 3 = 0\]
\[p(-1) = 3(-1)^3 - 5(-1)^2 - 11(-1) - 3 = -3 - 5 + 11 - 3 = 0\]
\[p\left(\frac{1}{3}\right) = 3\left(\frac{1}{3}\right)^3 - 5\left(\frac{1}{3}\right)^2 - 11\left(\frac{1}{3}\right) - 3 = \frac{1}{9} - \frac{5}{9} - \frac{11}{3} - 3 = 0\]

Therefore, $3$, $-1$, and $\frac{1}{3}$ are the zeroes of $3x^3 - 5x^2 - 11x - 3$. So we take:
\[\alpha = 3, \quad \beta = -1, \quad \gamma = \frac{1}{3}\]

Now we verify the relations:
\[\alpha + \beta + \gamma = 3 + (-1) + \frac{1}{3} = \frac{7}{3} = -\frac{b}{a} = -\frac{-5}{3}\]
\[\alpha\beta + \beta\gamma + \gamma\alpha = 3 \cdot (-1) + (-1) \cdot \frac{1}{3} + \frac{1}{3} \cdot 3 = -3 - \frac{1}{3} + 1 = -\frac{7}{3} = \frac{c}{a} = \frac{-11}{3}\]
\[\alpha\beta\gamma = 3 \cdot (-1) \cdot \frac{1}{3} = -1 = -\frac{d}{a} = -\frac{-3}{3}\]

Hence, the relationships between the zeroes and coefficients are verified.

\end{document}
