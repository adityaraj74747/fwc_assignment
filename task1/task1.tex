\documentclass[12pt]{article}
\usepackage[none]{hyphenat}
\usepackage{graphicx}
\usepackage{listings}
\usepackage[english]{babel}
\usepackage{caption}
\usepackage{hyperref}
\usepackage{booktabs}
\usepackage{array}
\usepackage{amsmath, amssymb}
\lstset{
  frame=single,
  breaklines=true
}
% New macro definitions
\newcommand{\mydet}[1]{\ensuremath{\begin{vmatrix}#1\end{vmatrix}}}
\providecommand{\brak}[1]{\ensuremath{\left(#1\right)}}
\providecommand{\norm}[1]{\left\lVert#1\right\rVert}
\newcommand{\solution}{\noindent \textbf{Solution: }}
\newcommand{\myvec}[1]{\ensuremath{\begin{pmatrix}#1\end{pmatrix}}}
\let\vec\mathbf

\begin{document}

\begin{center}
\textbf{\Large EXERCISE 2.1}
\end{center}

\vspace{1em}

\begin{figure}[h!]
\centering
\includegraphics[width=0.8\linewidth]{imgg.png}
\caption{Illustration related to polynomials}
\label{fig:poly_intro}
\end{figure}

\section*{2.3 Relationship between Zeroes and Coefficients of a Polynomial}

You have already seen that the zero of a linear polynomial \( ax + b \) is \( -\frac{b}{a} \). We will now try to answer the question raised in Section 2.1 regarding the relationship between zeroes
and coefficients of a quadratic polynomial. For this, let us take a quadratic polynomial, say
\( p(x) = 2x^2 - 8x + 6 \). In Class IX, you have learnt how to factorise quadratic polynomials by splitting the middle term. So, here we need to split the middle term ‘–8x’ as a sum of two terms, whose product is \(6 \times 2x^2 = 12x^2\). So, we write,

\textbf{Factorising:}
\[
\begin{aligned}
2x^2 - 8x + 6 &= 2x^2 - 6x - 2x + 6 \\
&= 2x(x - 3) - 2(x - 3) \\
&= (2x - 2)(x - 3) = 2(x - 1)(x - 3)
\end{aligned}
\]

So, the value of \( p(x) = 2x^2 - 8x + 6 \) is zero when \( x - 1 = 0 \) or \( x - 3 = 0 \), i.e., when
\( x = 1 \) or \( x = 3 \). So, the zeroes of \( 2x^2 - 8x + 6 \) are 1 and 3. Observe that:

\bigskip
\hspace*{6em} Sum of zeroes = \(1 + 3 = 4 = \frac{-(-8)}{2} = \frac{-\text{(Coefficient of } x)}{\text{Coefficient of } x^2}\)

\bigskip
\hspace*{6em} Product of zeroes = \(1 \cdot 3 = 3 = \frac{6}{2} = \frac{\text{Constant term}}{\text{Coefficient of } x^2}\)

\bigskip

Let us take one more quadratic polynomial, say, \( p(x) = 3x^2 + 5x - 2 \). By the method of splitting the middle term,
\[
\begin{aligned}
3x^2 + 5x - 2 &= 3x^2 + 6x - x - 2 \\
&= 3x(x + 2) - 1(x + 2) \\
&= (3x - 1)(x + 2)
\end{aligned}
\]

Hence, the value of \( 3x^2 + 5x - 2 \) is zero when either \( 3x - 1 = 0 \) or \( x + 2 = 0 \), i.e.,
when \( x = \frac{1}{3} \) or \( x = -2 \). So, the zeroes of \( 3x^2 + 5x - 2 \) are \( \frac{1}{3} \) and \( -2 \). Observe that:

\bigskip
\hspace*{6em} Sum of zeroes = \( \frac{1}{3} + (-2) = -\frac{5}{3} = \frac{-5}{3} = \frac{-\text{(Coefficient of } x)}{\text{Coefficient of } x^2} \)

\bigskip
\hspace*{6em} Product of zeroes = \( \frac{1}{3} \cdot (-2) = -\frac{2}{3} = \frac{\text{Constant term}}{\text{Coefficient of } x^2} \)

\bigskip

In general, if \( \alpha \) and \( \beta \) are the zeroes of the quadratic polynomial \( p(x) = ax^2 + bx + c \), \( a \neq 0 \), then \( x - \alpha \) and \( x - \beta \) are the factors of \( p(x) \). Therefore,

\[
ax^2 + bx + c = k(x - \alpha)(x - \beta), \quad \text{where } k \text{ is a constant}
\]
\[
= k[x^2 - (\alpha + \beta)x + \alpha\beta]
\]
\[
= kx^2 - k(\alpha + \beta)x + k\alpha\beta
\]

Comparing the coefficients of \( x^2, x \), and constant terms on both sides, we get:
\[
a = k, \quad b = -k(\alpha + \beta), \quad c = k\alpha\beta
\]

This gives:

\[
\alpha + \beta = \frac{-b}{a}, \quad \alpha \beta = \frac{c}{a}
\]

i.e.,

\[
\text{Sum of zeroes } = \alpha + \beta = \frac{-b}{a} = \frac{-\text{(Coefficient of } x)}{\text{Coefficient of } x^2}
\]

\[
\text{Product of zeroes } = \alpha \beta = \frac{c}{a} = \frac{\text{Constant term}}{\text{Coefficient of } x^2}
\]

Let us consider an example.

\end{document}

