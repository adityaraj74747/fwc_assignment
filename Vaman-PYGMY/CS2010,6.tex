\documentclass[12pt]{article}
\usepackage{amsmath}
\usepackage{array}
\usepackage{geometry}
\usepackage{setspace}
\geometry{margin=1in}
\renewcommand{\arraystretch}{1.4}
\setstretch{1.3}

\begin{document}

\section*{Minterm Expression of \(f(W, X, Y) = WX + X Y' + W Y'\)}

\subsection*{Step 1: Truth Table}

\[
\begin{array}{|c|c|c|c|c|c|c|c|}
\hline
W & X & Y & WX & XY' & WY' & f = WX + XY' + WY' \\
\hline
0 & 0 & 0 & 0 & 0 & 0 & 0 \\
0 & 0 & 1 & 0 & 0 & 0 & 0 \\
0 & 1 & 0 & 0 & 1 & 0 & 1 \\
0 & 1 & 1 & 0 & 0 & 0 & 0 \\
1 & 0 & 0 & 0 & 0 & 1 & 1 \\
1 & 0 & 1 & 0 & 0 & 0 & 0 \\
1 & 1 & 0 & 1 & 1 & 1 & 1 \\
1 & 1 & 1 & 1 & 0 & 0 & 1 \\
\hline
\end{array}
\]

\subsection*{Step 2: Identify Rows Where \(f = 1\)}

The function is true for the following combinations:
\[
\begin{aligned}
(0,1,0) &\Rightarrow m_2 = W' X Y' \\
(1,0,0) &\Rightarrow m_4 = W X' Y' \\
(1,1,0) &\Rightarrow m_6 = W X Y' \\
(1,1,1) &\Rightarrow m_7 = W X Y \\
\end{aligned}
\]

\subsection*{Step 3: Canonical SOP (Minterm) Expression}

\[
\boxed{f(W, X, Y) = m_2 + m_4 + m_6 + m_7}
\]

\section*{Vaman Board Pin Connections}

\begin{tabular}{|c|c|c|}
\hline
\textbf{Signal} & \textbf{PYGMY Pins} & \textbf{Description} \\
\hline
\(W\) & IO\_28 & Push button input with pull-down resistor \\
\(X\) & IO\_23 & Push button input with pull-down resistor \\
\(Y\) & IO\_31 & Push button input with pull-down resistor \\
\(f\) (Output) & 7-segment display & Active-low output to display \(f\) as digit \texttt{0} or \texttt{1} \\
\hline
\end{tabular}

\bigskip
\textit{Note: Connect push buttons between input pins and 3.3V.  
Enable internal pull-down resistors.  
Use a common cathode 7-segment display.  
Segment logic is active-low: segment glows when output is 0.}

\end{document}

